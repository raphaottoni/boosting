\section{Discussão e Conclusão}
Este trabalho prático foi interessante pois pude ver na prática a ideia de que se pode criar um classificador forte a partir de classificadores fracos. Mais do que isso, a técnica utilizada, \emph{Adaboost}, se provou muito resistente a  \emph{overfitting} e pode ser utilizada com qualquer tipo de classificadores. Não sendo limitada apenas a \emph{stumps}, como foi feito neste trabalho prático. 

Outra parte importante deste trabalho, foi comprovar conceitos sobre boosting aprendidos em sala de aula, como o porque é interessante apenas utilizar classificadores fracos no \emph{ensamble} e o porque idealmente eles deveriam ser um pouco melhores do que um classificador aleatório.
